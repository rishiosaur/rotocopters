\documentclass[]{article}
\usepackage{graphicx}	
\usepackage{amssymb}
\usepackage{amsmath}
\usepackage{subfiles}
\usepackage{geometry}
\usepackage{braket}
\usepackage{hyperref}
\hypersetup{
    colorlinks=true,
    linkcolor=black,
    filecolor=cyan,      
    urlcolor=cyan,
    pdftitle={Phi Design},
}

\usepackage{listings}
\usepackage{xcolor}

% \usepackage{arev}

% \usepackage{fontspec}
%     \setmainfont{SF Pro Text Regular}
%     \setmonofont{Fira Code Retina}

\usepackage{titlesec}
\titleformat{\chapter}[display]
  {\normalfont\huge}
  {\bfseries\chaptertitlename\ \thechapter}{20pt}{\Huge}

\titleformat{\section}
    {\normalfont\Large}
    {\thesection}{1em}{}

\titleformat{\subsection}
    {\normalfont\itshape\large}
    {\thesubsection}{1em}{}

\definecolor{codegreen}{rgb}{0.3,0.3,0.3}
\definecolor{codegray}{rgb}{0.5,0.5,0.5}
\definecolor{codepurple}{rgb}{0.69,0.69,0.69}
 
\lstdefinestyle{mystyle}{ 
    commentstyle=\color{codegreen},
    keywordstyle=\color{codegreen}\normalfont,
    numberstyle=\tiny\color{codegray},
    stringstyle=\color{codepurple},
    basicstyle=\ttfamily\footnotesize,
    breakatwhitespace=false,         
    breaklines=true,                 
    captionpos=b,                    
    keepspaces=true,                 
    numbers=left,                    
    numbersep=5pt,   
	tabsize=2,
    basicstyle=\ttfamily
}
 
\lstset{style=mystyle}


\usepackage{titling}
\renewcommand\maketitlehooka{\null\mbox{}\vfill}
\renewcommand\maketitlehookd{\vfill\null}

\title{\Huge The effects of weight distribution on flight time of rotocopter}
\author{\LARGE Your name here}
\date{}

\geometry{margin=1in}

\begin{document}
\pagenumbering{gobble}
\maketitle

\newpage

\tableofcontents
\newpage

\pagenumbering{arabic}

\section{Research Question}
How does altering the distribution of weight on a rotocopter's wings (between the left-to-right-wing-weight ratios ranging from $1$ to $>0.14$), measured in terms of grams, affect the total flight time of the rotocopter, measured in terms of seconds from the instant that the rotocopter is dropped from its initial height of 2.5m to the the time that the rotocopter touched the floor?

\section{Introduction}
One of the fundamental underpinnings of our world is the idea of flight; the world would not have become as accessible as it is today had there not been the breakthrough discoveries of thrust, lift, drag, and gravity. As our understanding of these factors changed, so too have our vehicles. Humans were able to engineer complex devices like the helicopter, airplane, and hybrids of these devices, and were able to change the ways that flight change. Focusing on the helicopter specifically, some of the properties that have been discovered to change the way that the object flies include the rounding of the wing, sharpness of a wing's rear edge, as well as wing tilt (). Each of these properties link back to the 4 main points, as discussed before, and are simply physical features that affect these mathematical factors. However, the experimenters' interests were piqued when they looked at an image of a helicopter's rotor blades. The blades seemed to be drooping on one side, but remained upright on the other, showing a difference in weight for both of the wings. Each of the aforementioned physical factors were found to affect flight purely through experimentation, and this idea was continued in this experiment.

In order to understand why helicopter blades drooped down, and explore how weight distribution affects flight, an adequate substitute needed to be found. The ideal substitute was found to be a rotocopter, for its striking relationship to a helicopter. In any flying vehicle, lift is always produced by air passing over some wing (). In a helicopter, the movement of the rotor blades allow for air to pass over them, and the pressure differential between the top of the wing and the bottom is what produces lift (). However, these rotor blades are able to produce lift because of their motorized capabilities; in having a motor, a greater speed can be achieved, which is eventually enough to combat the force of gravity. This allows the helicopter to manoeuvre in specialized ways, such as hovering, going straight up and down, rolling, and tilting forward. A rotocopter, on the other hand, does not have a motor, but it does have the same rotational energy as a helicopter, albeit at a much slower rate. As the wings of the rotocopter rotate, they, much like the helicopter, generate a small amount of lift. However, because of the rotocopter's lack of a motor, it merely combats the strength of gravity, and does not subdue it. This makes the rotocopter an excellent 

\section{Purpose}
The purpose of this experiment to understand the effects of weight distribution on flight time within the context of rotocopters. Using a total of 8 staples for each measurement and shifting the ratio of staples per wing will give a greater understanding of how weight \textit{placement} affects flight, as opposed to merely modulating the weight itself. Moving the metaphorical lens outwards, this experiment will also help develop a basic understanding of real-world flight factors, such as the mechanics behind airplane turning and helicopter lift.

\section{Hypothesis}
It is hypothesized that if the ratio of the weight of the left wing to the right wing of the rotocopter is decreased, then the time taken for the rotocopter to fly from its perch 2.5m above the ground to the ground will 

\subsection{Justification}
This experiment's testing procedure mimics the way that weight is distributed among the two wings of an aircraft.

When an aircraft needs to change directions, it cannot simply move along a vertical axis, as a car would. Instead, it 'banks', using two flaps 

\section{Variables}

\label{independent}
\subsection{Independent}
The only part of the experiment that is being changed is the 

\subsection{Dependent}
The total amount of time the rotocopter spent in the air, measured in seconds from the moment that the rotocopter was dropped from its perch $2.5 \pm 0.1$m above ground, to the point in time at which any part of the rotocopter was touching the chosen ground.

\subsection{Standardized}

\subsubsection{Height of Rotocopter perch}
The measured flight time is the total amount of time that it takes for a given rotocopter to travel from its perch at 

\subsubsection{Material choice and dimensions of Rotocopter}
Every material has a certain density associated with it. This means that as the material is changed, the mass of the given rotocopter also changes. According to Newton's Second Law of Motion (recall $F=ma$), a greater mass will result in a \textbf{greater time to accelerate to the same speed}. By extension, this means that a rotocopter with a larger mass would have an ending speed lower than that of a regular mass, thus increasing the flight time for the larger one, and skewing the data in the process\footnote{The control group was designed with this fact in mind}.

\subsubsection{Prescence of air drafts}
The measured (dependent) variable of the experiment, in a general context, is the time taken for an object to fly from one point to another as a result of some change to the given object. Introducing a headwind or tailwind (two types of air drafts) would change the speed of the object unnaturally, which would lead to a lower or higher speed, respectively. This would cause the timing to be skewed.

\subsubsection{Proximity to wall}
In order to measure the time taken for any object to fall, the object must be able to fall unobstructed. The proximity to any wall is especially important in an experiment of this nature -one dealing primarily with flight-, as flight paths are more often than not unpredictable.


\section{Materials}
\begin{itemize}
    \item Five (5) rotocopter stencils, printed out on standard white $8\frac{1}{2}$in x 11in paper
    \item One (1) pair of scissors
    \item One (1) meter stick, with increments in centimetres
    \item One (1) Stapler 
    \item Forty (40) Standard, 0.03$\pm$0.01g staples
    \item One (1) Machine thing u used 
    \item Timer (phone app) 
    \item One (1) DaVinci Resolve Video Editor
    \item One (1)
    \item One (1) Chair
    \item One (1) Assistant
\end{itemize}

\section{Procedure}
\begin{enumerate}
    \item Each of the rotocopter stencils were laid out on a table, and one was selected for usage.
    \item Using the pair of scissors, the stencil was cut according to its solid lines; the solid black lines served as guidelines for shaping the rotocopter’s net
    \item After cutting each of the stencils into the appropriate shape, crease each stencil along the dotted lines.
    \item This produced one (1) rotocopter, like so:
    \item Image
    \item For each of the two ‘wings’ of the rotocopter, marks were made 2±0.05cm from the centre of the ending edge of each, as seen in Figure 2:
    \item Steps 1-3 were repeated for each of the other 4 templates.
    \item One of the rotocopters were designated to be a part of the control group, and have a ratio of 0.12±0.03g:0.12±0.03g, or 4 staples on the right wing, and 4 staples on the left wing.
    \item For this control group rotocopter, hold the centre part of the copter such that your hand is clamped on the bridge between one wing and the tail of the rotocopter.
    \item Staple 4 0.03$\pm$0.01g staples on the wing that is facing away from you directly on the mark mentioned in step 4.
    \item Turn the 
    \item Each of the other 4 rotocopters were designated to be modified as per the requirements set out by the independent variable (See \nameref{independent})
    \item One of the rotocopters were selected, and 
    \item Steps 8-10 for each of the other rotocopters.
    \item Set each of the rotocopters down on a flat surface, such that they are all facing upwards.
    \item A space of 3m x 3m x 2.5m, with at least one wall was designated for experimentation.
    \item The metre stick was aligned with the wall, such that the stick was parallel to the wall, while still touching it.
    \item The alignment was a crucial step in the process, as this is what designated the drop of the 
    \item Mark the Measurement with a piece of tape(the bottom of the tape must be at the end of 2.5m measurement) 
    \item Move the chair next to the wall
    \item Make sure that the chair is facing 90$\deg$ away from the wall(assuming 0$\deg$ is facing the wall)
    \item Stand on the chair(make sure to have the back post behind you)
    \item Make sure to have a rotocopter with one adjustment(the ratio of staples on each wing)
    \item Hold the rotocopter so that the bottom of the stem is at where the tape ends(the bottom part of the tape)
    \item Make sure to have your partner ready to time/record the drop (with their app or stopwatch)
    \item Drop the rotocopter after the count of 3..2..1...Go
    \item At Go make sure that your partner starts the timer (this is when you drop the rotocopter at Go)
    \item Record the results (the time)
    \item Repeat his process 5 more times to ensure an accurate measurement (for a total of 6 tests)
\end{enumerate}


\section{Data}

\subsection{Uncertainties}
The experimental values were calculated using a variety of factors, but all had one thing in common: the interactions of humans. In recording the data, humans were employed for the tast of starting and ending the recording, which 

\section{Discussion}

\subsection{Data}
It was initially hypothesized that as the weight distribution ratio between the wings was decreased, the flight time would also decrease. In other words, plotting the average flight time of each trial against the ratio of the wing-weight distribution should reveal a negative, linear correlation. This hypothesis was partially supported in the graphs; in Figure 4, it can be seen that the line of best fit (hereon referred to as the LOBF) followed this pattern of a decreasing negative correlation. The data in both Table 1 and Graph 1 showcase this fact, as each of the times are decreasing in correlation with a lower wing-weight ratio. However, the $R^2$ value of the LOBF in Table 1 was about $0.3$, which means that it had an accuracy measurement of $\pm 30\%$. This was an unacceptable value to draw conclusions from, thus necessitating an adjustment to the LOBF. The experimenters changed the modelling equation from a linear model to a polymnomial one of degree $2$, as there was a dip in the data around the $2:6$ ratio measurement. After this was done, the $R^2$ value jumped up to $0.95$, meaning that it was a much more accurate model, as can be seen by the closeness of the points to the curve in Figure 5.

Because of the jump in the $R^2$ value with the second curve, the polynomial curve was deemed to be the best representation of the data at hand. Analyzing this curve, one can see that there is an 'optimal value' of the graph, whose y-value represents the time taken for the rotocopter to reach the ground, and the x-value representing the ratio of the weight of the left wing to the weight of the right. Coming back to the main point, this graph and optimized data only partially proves the hypothesis. it was stated that there would be a negative \textit{linear} correlation. The graph proves that there is a point at which the speed simply cannot be increased anymore; in other words, there is a minimum value, which is an idea that is lacking in the linear prediction. However, the graph also shows that there is a general decrease in the times, which confims the first part of the hypothesis, which states that there is a general decrease in the flight times, as the weight ratio between wings decreases.

\subsection{Possible Errors}
Despite the data being quite precise (the ranges in Figure 4 are all relatively small), there could still have been both systematic and random errors made in the collection process, which might have skewed the data.

\subsubsection{Systematic}

One of the major factors affecting flight is wind and humidity, and this affects flight time readings for small objects much more; as a result of their small mass, objects like paper rotocopters are much more finicky. Although the experiment was conducted in a section of a hallway, 

\subsubsection{Random}

\subsection{Improvements}

\section{Conclusion}
In conclusion, the purpose of this experiment was to understand the effects of weight distribution of the wings of a rotocopter, between ratios ranging from $1$ to $>0.14$, on the time taken to fly from one point 2.5m above the ground to the ground. This was done by stapling a total of 8 staples to the wings of the rotocopters, 2$\pm$0.05cm away from a given wing's edge, with each trial having a different ratio of staples 

As is shown in the graph, the data collected follows the pattern of 



\end{document}